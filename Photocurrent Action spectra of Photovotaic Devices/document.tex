%%This is a very basic article template.
%%There is just one section and two subsections.
\documentclass{article}
\usepackage{graphicx}
\usepackage{amsmath}
\usepackage{color}
\begin{document}
\title{Photocurrent Action spectra of Photovotaic Devices}
\maketitle
\section{Internal Optical electric field and energy dissipation due to an incident plane wave
}

\subsection{Transfer Matrix Method:~\cite{pettersson1999modeling}}

Stratified structures can be described by $2\times2$ matrix 
due to the fact that the equation governing the propagation 
of the electric field are linear and that the tangential component 
of the electric field is continuous.Consider a plane wave incident from left at a general multilayer 
structure having m layers between a semi-infinite transparent ambient 
and a semi-infinite substrate as schematically described in Figure
\ref{fig:layers}.
\begin{figure}[h!]
  \centering
    \includegraphics[width=0.7\textwidth]{layers}
  \caption{A general multilayer structure having m layers between a semi 
  infinite transparent ambient and a semi-infinite substrate. 
 Each layer j(j=1,2,…,m) has a thickness $d_{j}$ and its optical properties are
 described by its complex index of refraction. The optical electric field at any
 point in lyaer j is represented by two components: one propagating in the
 positive and one in the negative x direction, $\mathbf{E_{j}^{+}}$ and
 $\mathbf{E_{j}^{-}}$, respectively} \label{fig:layers} 
\end{figure}
\subsubsection{Theoretical Background}
The electric field in layer j can be expressed in terms of the matrix elements
of the partial system transfer matrices as

\begin{equation}
{{E}_{j}}(x)=\frac{S_{j11}^{''}\cdot {{e}^{-i{{\xi
}_{j}}({{d}_{j}}-x)}}+S_{j21}^{''}\cdot {{e}^{i{{\xi }_{j}}({{d}_{j}}-x)}}}{S_{j11}^{'}S_{j11}^{''}\cdot {{e}^{-i{{\xi }_{j}}{{d}_{j}}}}+S_{j12}^{'}S_{j21}^{''}\cdot {{e}^{i{{\xi }_{j}}{{d}_{j}}}}}E_{0}^{+}
\label{mainequation} 
\end{equation}
In Equation \ref{mainequation}:

$d_j$ is the thickness of the jth layer as shown in Figure \ref{fig:layers}.
\begin{equation}
{{\xi }_{j}}=\frac{2\pi }{\lambda }{{q}_{j}}
\end{equation}
\begin{equation}
S_{j}^{'}=\left[ \begin{matrix}
   S_{j11}^{'} & S_{j12}^{'}\\
   S_{j21}^{'} & S_{j22}^{'}  
\end{matrix} \right]=\left( \prod\limits_{v=1}^{j-1}{{{I}_{(v-1)v}}{{L}_{v}}} \right)\cdot {{I}_{(j-1)j}}
\label{Sp} 
\end{equation}
\begin{equation}
S_{j}^{''}=\left[ \begin{matrix}
   S_{j11}^{''} & S_{j12}^{''}\\
   S_{j21}^{''} & S_{j22}^{''}\\
\end{matrix} \right]=\left( \prod\limits_{v=j+1}^{m}{{{I}_{(v-1)v}}{{L}_{v}}} \right)\cdot {{I}_{m(m+1)}}
\label{Spp} 
\end{equation}
\begin{equation}
{{I}_{jk}}=\frac{1}{{{t}_{jk}}}\left[ \begin{matrix}
   1 & {{r}_{jk}}  \\
   {{r}_{jk}} & 1  \\
\end{matrix} \right]
\end{equation}
\begin{equation}
{{L}_{j}}=\left[ \begin{matrix}
   {{e}^{-i{{\xi }_{j}}{{d}_{j}}}} & 0  \\
   0 & {{e}^{i{{\xi }_{j}}{{d}_{j}}}}  \\
\end{matrix} \right]
\end{equation}
For light with the electric field perpendicular to the plane of
incidence(s-polarized or TE waves), the Fresnel complex reflection and
transmission coefficients are defined by
\begin{subequations}
\begin{equation}
{{r}_{jk}}=\frac{{{q}_{j}}-{{q}_{k}}}{{{q}_{j}}+{{q}_{k}}}
\end{equation}
\begin{equation}
{{t}_{jk}}=\frac{2{{q}_{j}}}{{{q}_{j}}+{{q}_{k}}}
\end{equation}
\end{subequations}
and for light with the electric field parallel to the plane of
incidence(p-polarized or TM waves) as
\begin{subequations}
\begin{equation}
{{r}_{jk}}=\frac{\tilde{n}_{k}^{2}{{q}_{j}}-\tilde{n}_{j}^{2}{{q}_{k}}}{\tilde{n}_{k}^{2}{{q}_{j}}+\tilde{n}_{j}^{2}{{q}_{k}}}
\end{equation}
\begin{equation}
{{t}_{jk}}=\frac{2{{{\tilde{n}}}_{j}}{{{\tilde{n}}}_{k}}{{q}_{j}}}{\tilde{n}_{k}^{2}{{q}_{j}}+\tilde{n}_{j}^{2}{{q}_{k}}}
\end{equation}
\end{subequations}
where
\begin{equation}
{{q}_{j}}={{\tilde{n}}_{j}}\cos {{\phi }_{j}}={{[\tilde{n}_{j}^{2}-\eta
_{0}^{2}\sin {{\phi }_{0}}]}^{1/2}}
\end{equation}
\subsubsection{IOEF(Internal Optical Electric Field) Class Reference}
This class is written in Python 2.7.6, you can use this class to obtain internal
optical electric field due to an oncident plane wave
within the structure you defined in the class attributes.

\textbf{Instance Attributes:(id, phi0, wavelength, TME)}

\textcolor{blue}{id:} Define the Structure with a numpy.array Type.

\textcolor{blue}{phi0:} Define the incidence angle.

\textcolor{blue}{wavelength:} Define the wavelength.

\textcolor{blue}{TME:} Define the plorization of light (TE or TM).

\textbf{Method:}

\textcolor{blue}{EFfunc(j,x):} Get the electric field distributon at x in layer
j

\textcolor{blue}{plotEFfunc(layer):}Plot the electric field distribution vs x in
layer j










\subsection{Another subtitle}
More plain text.

\bibliography{mybib}{}
\bibliographystyle{plain}
\end{document}
